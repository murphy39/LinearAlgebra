% Example LaTeX document for GP111 - note % sign indicates a comment
\documentclass[12pt]{amsart}
\usepackage[top=1.5cm, left=1.5cm,right=1.5cm,bottom=1.5cm]{geometry}

% \addtolength{\hoffset}{-2.5cm}
% \addtolength{\textwidth}{5cm}

% \addtolength{\voffset}{-1.25cm}
% \addtolength{\textheight}{2.5cm}

\usepackage{amsmath}
\usepackage{amssymb}
\usepackage{color,hyperref}
\definecolor{darkblue}{rgb}{0.0,0.0,0.3}
\hypersetup{colorlinks,breaklinks,
            linkcolor=darkblue,urlcolor=darkblue,
            anchorcolor=darkblue,citecolor=darkblue}
            
\pagestyle{empty}

\begin{document}
\thispagestyle{empty}

\begin{center} \textbf{Math 700 -- Linear Algebra -- Spring 2014}

 \end{center}
\vskip5mm

\noindent {\bf Instructor:} Dr.~William DeMeo  \hfill {Phone: 777-7510}\\
~\phantom{XX} \hfill {Email: \href{mailto:williamdemeo@gmail.com}{williamdemeo@gmail.com}}

\noindent {\bf Office:} LeConte College, Room 314C \\

\noindent {\bf Office Hours:} Wednesday, Thursday, Friday 11--12, and by appointment\\

%\noindent {\bf Prerequisites:} Grades of C or better in both CSCE 146 and MATH 142.\\

\noindent {\bf Main Textbook:} Peter Lax, \emph{Linear Algebra and Its Applications}, Wiley, New York, NY, 2007. \href{http://www.whfreeman.com/gersting}{www.whfreeman.com/gersting}. \\
\noindent {\bf Other References:} Peter Lax, \emph{Linear Algebra and Its Applications}, Wiley, New York, NY, 2007. \href{http://www.whfreeman.com/gersting}{www.whfreeman.com/gersting}. \\

\noindent {\bf Topics:}
<ol><li><b>Review of Fundamentals:</b>
Linear Space, Isomorphism; Subspace; Linear Dependence; Basis, Dimension; Quotient Space
</li>
<li><b>Duality:</b> Linear Functions; Dual of a Linear Space; Annihilator; Codimension; Quadrature Formula.</li>
</li>
<li><b>Linear Mappings:</b> Domain and Target Space; Nullspace and Range; Fundamental Theorem; Underdetermined Linear Systems; Interpolation; Difference Equations; Algebra of Linear Mappings; Dimension of Nullspace and Range; Transposition; Similarity; Projections.
</li>
<li><b>Matrices:</b> Rows and Columns, Matrix Multiplication, Transposition, Rank, Gaussian Elimination
</li>
<li><b>Determinant and Trace:</b> Ordered Simplices; Signed Volume,  Determinant; Permutation Group; Formula for Determinant; Multiplicative Property; Laplace Expansion; Cramer's Rule; Trace</li>
<li><b>Spectral Theory:</b> Iteration of Linear Maps; Eigenvalues, Eigenvectors; Fibonacci Sequence; Characteristic Polynomial; Trace and Determinant Revisited; Spectral Mapping Theorem; Cayley-Hamilton Theorem; Generalized Eigenvectors; Spectral Theorem; Minimal Polynomial; When Are Two Matrices Similar; Commuting Maps.</li>
<li><b>Euclidean Structure:</b> Scalar Product, Distance; Schwarz Inequality; Orthonormal Basis; Gram-Schmidt; Orthogonal Complement; Orthogonal Projection; Adjoint; Overdetermined Systems; Isometry; The Orthogonal Group; Norm of a Linear Map; Completeness Local Compactness; Complex Euclidean Structure; Spectral Radius; Hilbert-Schmidt Norm; Cross Product.
</li>
<li><b>Spectral Theory of Self-Adjoint Mappings:</b> Quadratic Forms; Law of Inertia; Spectral Resolution; Commuting Maps; Anti-Self-Adjoint Maps; Normal Maps; Rayleigh Quotient; Minmax Principle; Norm and Eigenvalues.
</li>
<li><b>
Calculus of Vector- and Matrix-Valued Functions:</b> Convergence in Norm; Rules of Differentiation, Derivative of det A(t), Matrix Exponential; Simple Eigenvalues; Multiple Eigenvalues; Rellich's Theorem; Avoidance of Crossing.
</li>
<li><b>Matrix Inequalities:</b> Positive Self-Adjoint Matrices;
Monotone Matrix Functions;
Gram Matrices;
Schur's Theorem;
The Determinant of Positive Matrices;
Integral Formula for Determinants;
Eigenvalues;
Separation of Eigenvalues;
Wielandt-Hoffman Theorem;
Smallest and Largest Eigenvalue;
Matrices with Positive Se]f-Adjoint Part;
Polar Decomposition;
Singular Values; 
Singular Value Decomposition; 
</li>
</ol>
And a selection of topic from the following, as time permits:
<ul>
<li><b>Convexity:</b>
Convex Sets;
Gauge Function;
Hahn-Banach Theorem;
Support Function;
Caratheodory's Theorem;
Konig-Birkhoff Theorem;
Helly's Theorem.
</li>
<li><b>
The Duality Theorem:</b>
Farkas-Minkowski Theorem;
Duality Theorem;
Economics Interpretation;
Minmax Theorem.
</li>
<li><b>
Normed Linear Spaces:</b>
Norm;
l<sup>p</sup> Norms;
Equivalence of Norms;
Completeness; 
Local Compactness;
Theorem of F. Riesz;
Dual Norm; 
Distance from Subspace; 
Normed Quotient Space; 
Complex Normed Spaces; 
Complex Hahn-Banach Theorem; 
Characterization of Euclidean Spaces. 
</li>
<li><b>
<b>Linear Mappings Between Normed Linear Spaces:</b>
Norm of a Mapping;
Norm of Transpose;
Normed Algebra of Maps;
Invertible Maps;
Spectral Radius;
</li>
<li><b>
Positive Matrices:</b>
Perron's Theorem;
Stochastic Matrices;
Frobenius' Theorem;
</li>
<li><b>
How to Calculate the Eigenvalues of Self-Adjoint Matrices:</b>
QR Factorization;
Using the QR Factorization to Solve Systems of Equations;
The QR Algorithm for Finding Eigenvalues; 
Householder Reflection for OR Factorization;
Tridiagonal Form; 
Analogy of QR Algorithm and Toda Flow; 
Moser's Theorem.
</li>
</ol>

\noindent {\bf Class Meetings:} MW, 3:55--5:10, LeConte, Room 310. \\

\noindent {\bf Exams:} There will be one mid-term exam during the semester, and will be scheduled as we progress. Notice of at least one week will be given prior to the date of a
test. A comprehensive final exam will be held on .\\
 
\noindent {\bf Homework:} Solving homework problems is the easiest, best, and perhaps only way
to properly prepare for tests and quizzes.   Homework will be assigned and
collected regularly, and selected problems from each set will be graded for
correctness. Solutions are due by the end of class on the deadline date
assigned, or can be left at my office \emph{earlier}. Late homework will not be
accepted or graded. Your best ten homework assignments will be counted toward
your grade. \\ 

\noindent {\bf Quizzes:} We will have two quizzes, one after Chapter 1, covering
material from that chapter, and one after Chapter 3, covering the material from
that chapter.  \\

\noindent {\bf Make-up Policy:} \emph{There will be no make-up homework or quizzes for any
  reason.} If you must miss class for a legitimate reason, contact me prior to
the class and we may be able to schedule a time \emph{before} the date in
question. If you miss a test with a legitimate reason, you may come to me within
one class meeting of the missed test and provide an explanation, in which case I might
agree to replace the missed test score with 80\% of your final exam score.
For example, if I accept your excuse and you score 90\% on the final, then I
will give you 72\% on the missed test (0.80*0.90 = 0.72).\\

\noindent {\bf Grading Scheme:} Each test will be worth 25 points, the final
will be worth 30 points, each quiz will be worth 5 points, and homework will be worth 10 points, 
The following scores will guarantee the following grades. I reserve the right to shift the scale at the end of the semester. All curving (if any) will occur at the end of the semester.

$$ \begin{array}{ll}
A&:  91 - 100 \\
B+&: 87 - 91 \\
B &: 81 - 87 \\
C+&: 77 - 81 \\
C &: 71 - 77 \\
D+ &: 67 - 71 \\ 
D &: 60 - 67 \\
F &: 0 -60 
\end{array}$$ \\
    
\noindent {\bf Course Website:}
\url{http://williamdemeo.wordpress.com/teaching/math-374-fall-2013/}\\
The course webpage will list the material covered, the assigned homework, test dates, and any
supplemental material. It will be updated often as we progress through the
material, so you should check it routinely.\\ 

\noindent {\bf Course Content:} We will cover as much as time permits of the
following topics: propositional and predicate logic; proof techniques; recursion
and recurrence relations; sets and combinatorics; functions, relations and
matrices, lattices, graphs and trees. Logic, proof techniques, and recursion provide the
mathematical foundation for both writing a program and demonstrating its
correctness. Sets, combinatorics, functions, relations, matrices and lattices comprise
the most basic objects and relationships that are used in computer
science. Graphs and trees can be used as models of many real world pheneomena,
while also being amenable to computer programming. \\ 

\noindent {\bf Learning Outcomes:} Students are expected to translate English
sentences into predicates and vice versa. They will be able to evaluate truth
values and verify tautologies using methods of logic and will be familiar with
the principles of declarative programming languages. 

Students will be able to write recursive algorithms, prove the correctness of
simple algorithms, solve simple recurrences, and use mathematical induction, in
particular, to show the correctness of loops. 

Students will be able to use the concepts of relations, graphs and matrices to
model real-life situations and to operate within the models they created. 
  
Students are, of course, expected to read the text after class. However, reading
the next section \emph{before} class is also expected. This will help the
students in seeing the outline of the current topic, giving familiarity with the
basic concepts, and help in seeing the fine points of the lecture.\\ 


\noindent {\bf Academic Honesty:} Cheating will not be tolerated in this
course. Violations of this policy will be referred to and dealt with by the USC
Office of Academic Integrity, in a  manner consistent with university
regulations, which range from a warning to expulsion from the university. \\

\noindent {\bf Electronics policy:} Please silence and refrain from using all
electronic devices (phones, ipods, tablets, microwave ovens, etc.)  during class and exam periods.
\\

\noindent {\bf Some Important Dates:}\\
Aug 22, Thu:	Classes begin\\
Aug 28, Wed:	Last day to drop/add without a grade of “W” being recorded\\
Sep 2, Mon:	Labor Day Holiday - no classes\\
Sep 12, Thu:	Last day to apply for December graduation\\
Oct 11, Fri:	Last day to drop a course or withdraw without a grade of “WF” being recorded\\
Oct 17--18, Thu--Fri:	Fall break-no classes\\
Nov 27--Dec 1, Wed--Sun:  Thanksgiving recess-no classes\\
Dec 6, Fri: Last day of classes\\
Dec 7, Sat: Reading day\\
Dec 9--16, Mon--Mon:	Final Examinations.










\end{document}
